\documentclass[12pt]{article}
\usepackage[spanish]{babel}
\usepackage[utf8]{inputenc}
\usepackage{graphicx}
\usepackage{xcolor}
\usepackage{geometry}
\usepackage{amsmath}
\usepackage[section]{placeins}
\usepackage{hyperref}
\hypersetup{
    colorlinks = true,
    linkcolor = blue,
    urlcolor = cyan
}

\geometry{a4paper, margin=2.5cm}

\title{Documentación Práctica Hilos Coches - Interfaz}
\author{Daniel Calvar Cruz}
\date{\today}

\begin{document}

\maketitle

\tableofcontents
\newpage

\section{Introducción}
\hyperlink{anchor-indice}{\textbf{Volver}}\\

Implementación de interfaz gráfica al ejercicio de carreras de coches con uso de threads, con el añadido de Synchronize.

\section{Estructura}
\hyperlink{anchor-indice}{\textbf{Volver}}\\

La aplicación está estructurada siguiendo el sistema MVC (Modelo-Vista-Controlador), que proporciona una clara separación de responsabilidades entre la interfaz gráfica, la lógica del negocio y los datos.


CarreraApplication lanza \verb|carrera-view.fxml|, controlado por \verb|CarreraController|, que se ocupa de toda la lógica de negocio de la carrera en si, comienzo, fin y funcionamiento. Este controlador usa las clases:
\begin{itemize}
  \item Carrera: crea un hilo para cada carrera. Muestra el podio al finalizar.
  \item Coche: crea los hilos de Coche usados para cada carrera, logica para las vueltas y control de UI.
\end{itemize}

\section{Synchronize}
\hyperlink{anchor-indice}{\textbf{Volver}}\\

Usamos \verb|Synchronize| en:

\begin{itemize}
  \item \verb|agregarCoche| (Carrera): para agregar los coches que han finalizado la carrera a la lista que se emplea en la creación del podio.
  \item Mecanica de aceleracion (Coche): mecánica de aceleración que controla la distancia recorrida de cada coche.
  \begin{verbatim}
    // mecanica de aceleracion
            synchronized (this) {
                int aceleracion = (int)(Math.random() * 10) + 1;

                // para inicio de carrera o funcionamiento normal
                if (velocidadActual == 0 || velocidadActual < velocidadMaxima) {
                    velocidadActual += aceleracion;
                    distanciaRecorridaEnVuelta += velocidadActual;
                }

                // penalizacion para sobrepasado de velocidad maxima
                if (velocidadActual >= velocidadMaxima) {
                    velocidadActual -= aceleracion * 2;
                    distanciaRecorridaEnVuelta += velocidadActual;
                }
            }
  \end{verbatim}
  \item Verificar vueltas (Coche): verifica la "posición" del coche para saber cuando completa una vuelta:
  \begin{verbatim}
    // verificar vueltas
            synchronized (this) {
                if (distanciaRecorridaEnVuelta > (distanciaVuelta / 2) && mitadVueltaFlag) {
                    mitadVueltaFlag = false;
                }

                if (distanciaRecorridaEnVuelta >= distanciaVuelta) {
                    vueltaActual++;
                    distanciaRecorridaEnVuelta = 0;
                    mitadVueltaFlag = true;

                    // actualizar pogreso
                    if (controller != null && controller.isCarreraEnCurso()) {
                        controller.actualizarProgresoGeneral();
                    }

                    Platform.runLater(() -> {
                        if (!isInterrupted()) {
                            estadoLabel.setText("Vuelta " + (vueltaActual + 1) + "/" + numeroVueltas);
                        }
                    });
                }
            }
  \end{verbatim}
  \item \verb|actualizarProgresoGeneral| (CarreraController): actualización de la barra de progreso general cada vez que un coche completa una vuelta.
\end{itemize}

\section{Interfaz}
\hyperlink{anchor-indice}{\textbf{Volver}}\\

El diseño se estructura en secciones claramente definidas que facilitan la comprensión y el uso del sistema.

Estructura General:

La interfaz se organiza en un contenedor principal VBox que alberga todas las secciones, proporcionando un flujo visual descendente claro e intuitivo.

\begin{itemize}
    \item Encabezado
    \item Pistas
    \item Podio (oculto hasta final de carrera)
    \item Botonera
\end{itemize}

Características Técnicas
\begin{itemize}
    \item Actualización en tiempo 'real' de estados
    \item Interfaz intuitiva y fácil de usar
\end{itemize}

\section{Requisitos y tecnologías}
\hyperlink{anchor-indice}{\textbf{Volver}}\\

Dependencias:
\begin{itemize}
    \item Maven v4.0.0
    \item Junit v5.12.1
    \item javafx-controls v21.0.6
    \item javafx-fxml v21.0.6
    \item jupiter-api v5.12.1
    \item jupiter-engine v5.12.1
\end{itemize}

Plugins:
\begin{itemize}
    \item maven-compiler-plugin v3.13.0
    \item javafx-maven-plugin v0.0.8
\end{itemize}

IDE:
\begin{itemize}
    \item IntelliJ IDEA 2025.2.1 (Community Edition)
    \item Build IC-252.25557.131, built on August 27, 2025
    \item Source revision: ee1e6cb62e111
    \item Runtime version: 21.0.8+1-b1038.68 amd64 (JCEF 122.1.9)
    \item VM: OpenJDK 64-Bit Server VM by JetBrains s.r.o.
    \item Toolkit: sun.awt.windows.WToolkit
\end{itemize}


\section{Recursos gráficos}
\hyperlink{anchor-indice}{\textbf{Volver}}\\

Icono banderitas de carreras: \url{https://www.flaticon.com/free-icons/start}, autor: Smashicons - Flaticon.
\end{document}